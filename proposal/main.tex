\documentclass[
  12pt,
  openright,
  twoside,
  a4paper,
  english,
  brazil
]{abntex2}

\usepackage[T1]{fontenc}
\usepackage{indentfirst}
\usepackage{pdfpages}
\usepackage[alf]{abntex2cite}
\usepackage{booktabs}
\usepackage{pifont}

\setlength{\parindent}{1.3cm}

\titulo{Implementação de um front end para \textit{LLVM intermediate representation}}

\autor{Bernardo Ferrari Mendonça}
\local{Brasil}
\data{2019}
\orientador{Rafael de Santiago}
\coorientador{Evandro Chagas Ribeiro da Costa}
\instituicao{
  Universidade Federal de Santa Catarina
  \par
  Departamento de Informática e Estatística
  \par
  Ciência da Computação
}

\tipotrabalho{Trabalho de Conclusão de Curso de Graduação}

\preambulo{Proposta de monografia submetida ao Programa
de Graduação em Ciência da Computação
para a obtenção do Grau de Bacharel.}

\begin{document}
\pretextual{}
\hypersetup{pageanchor=false}
\imprimircapa{}
\hypersetup{pageanchor=true}
\imprimirfolhaderosto{}

\begin{folhadeaprovacao}
  \includepdf{folha-de-aprovacao.pdf}
\end{folhadeaprovacao}

\begin{resumo}
Nos dias atuais todos os programas de computadores são descritos em alguma linguagem de programação.
Compiladores são os programas de computadores que fazem o papel de tradutor entre as linguagens de programação e as linguagens de maquina.
Eles recebem como entrada um texto escrito em uma linguagem fonte, chamado de programa fonte, e produzem como saída um texto escrito em uma linguagem de maquina alvo, chamado de programa alvo.

O processo de tradução executado por um compilador pode ser separado em duas grandes etapas: a etapa de analise e a etapa de síntese.
Entre elas existe uma linguagem intermediaria.
O uso de uma linguagem intermediaria é especialmente vantajoso pois, um compilador para a linguagem $i$ e maquina $j$ pode ser construído combinando um \textit{front end} para $i$ e um \textit{back end} para $j$.

O projeto LLVM é uma biblioteca de funcionalidades para otimização de código intermediário e geração de código alvo.
Estas funcionalidades são construídas ao redor de uma linguagem intermediaria chamada \textit{LLVM intermediate representation}, ou LLVM IR\@.

\vspace{\onelineskip}
\noindent
\textbf{Palavras-chave}: linguagem de programação\@. representação intermediaria de código\@. compilador\@. LLVM\@. LLVM IR\@.

\end{resumo}

\begin{KeepFromToc}
  \tableofcontents
\end{KeepFromToc}

\textual{}

\chapter{Introdução}\label{cap:introducao}

Linguagens de programação são notações para descrever computações para pessoas e para máquinas.
Nos dias atuais todos os programas de computadores são descritos em alguma linguagem de programação
pois facilitam a comunicação das suas computações entre as pessoas que os desenvolvem.
Porém, antes que estas computações possam ser executadas por um computador,
elas necessitam ser traduzidas para uma linguagem de maquina passível de ser executada por um computador.

Compiladores são os programas de computadores que fazem o papel de tradutor entre as linguagens de programação e as linguagens de maquina.
Eles recebem como entrada um texto escrito em uma linguagem fonte, chamado de programa fonte, e produzem como saída um texto escrito em uma linguagem de maquina alvo, chamado de programa alvo.
Para que consigam realizar essa tradução, compiladores utilizam de modelos formais da teoria da computação e de teoria de linguagens formais.

\section{O processo de tradução}\label{cap:introducao:sec:o_processo_de_traducao}

O processo de tradução executado por um compilador pode ser separado em duas grandes etapas: a etapa de analise e a etapa de síntese.
Essas etapas podem ser mais uma vez subdivididas.

A etapa de analise, ou \textit{front end} de um compilador, pode ser subdividida em: analise léxica; analise sintática; analise semântica; e geração de código intermediário.
Esta etapa é explorada na seção~\ref{cap:introducao:sec:o_front_end}.

Além disso, a etapa de síntese, ou \textit{back end} de um compilador, pode ser subdividida em: otimização de código independente de maquina; geração de código alvo; e otimização de código dependente de maquina.
Esta etapa é explorada na seção~\ref{cap:introducao:sec:o_back_end}.

O uso de uma linguagem intermediaria é especialmente vantajoso pois, um compilador para a linguagem $i$ e maquina $j$ pode ser construído combinando um \textit{front end} para $i$ e um \textit{back end} para $j$.
Além disso, se desejarmos implementar $n$ linguagens de programação para $m$ maquinas, evitamos a construção de $n \times m$ compiladores construindo $n$ \textit{front ends} e $m$ \textit{back ends}.

\subsection{O \textit{front end}}\label{cap:introducao:sec:o_front_end}

A sub etapas de um \textit{front end} são descritas em sequencia.

A analise léxica consiste em identificar os lexemas presentes no código fonte e produzir \textit{tokens} para cada um dos lexemas encontrados.
Estruturas formais como expressões regulares e autómatos finitos, juntamente com um algorítimo de conversão~\cite{lesk1975lex}, podem ser utilizadas para geração automática de analisadores léxicos.

A analise sintática consiste em analisar os \textit{tokens} produzidos pelo analisador léxico de produzir uma arvore sintática.
Estruturas formais como linguagens livre de contexto e maquinas de pilha, juntamente com um algorítimo de \textit{parsing}~\cite{knuth1965translation}, podem ser utilizadas para geração automática de analisadores sintáticos.

A analise semântica consiste em analisar a arvore sintática produzida pelo analisador sintático e produzir código intermediário.
Estruturas formais como \textit{Syntax Directed Definitions} (SDD), juntamente a uma \textit{Syntax Directed Translation} (SDT), podem ser utilizadas para geração de código intermediário~\cite{Aho:2006:CPT:1177220}.

\subsection{O \textit{back end} e a linguagem intermediaria LLVM}\label{cap:introducao:sec:o_back_end}

O \textit{back end} de um compilador realiza a etapa de síntese e é uma area muito importante de compiladores.
É o \textit{back end} que realiza as otimizações necessárias para que os programas de computador sejam eficientes.

O projeto LLVM~\cite{lattner2004llvm} é uma biblioteca de funcionalidades para otimização de código intermediário e geração de código alvo.
Estas funcionalidades são construídas ao redor de uma linguagem intermediaria chamada \textit{LLVM intermediate representation}, ou LLVM IR\@.

\chapter{Objetivos}\label{cap:objetivos}

O principal objetivo deste trabalho consiste em projetar, implementar e avaliar uma linguagem de programação similar a linguagem de programação c.
Esta implementação de linguagem de programação deve fazer uso do ecossistema llvm, incluindo suas ferramentas para geração de linguagem de representação intermediaria, para otimização de código e para geração de código objeto executável.

\section{Objetivos específicos}

Os objetivos específicos são:
\begin{alineas}
  \item projetar uma linguagem de programação similar a c;
  \item implementar um analisador léxico, um analisador sintático e uma tradução dirigida a sintaxe;
  \item fazer um estudo sobre a linguagem intermediaria llvm;
  \item implementar uma ferramenta para controlar a otimização da linguagem intermediaria llvm. Além de controlar a tradução da mesma para código objeto;
  \item fazer um estudo sobre o processo de ligação de código;
  \item integrar a ferramenta ao processo de ligação de código do ambiente GNU/Linux;
  \item avaliar as decisões de projeto da linguagem de programação projetada quanto a linguagem c.
\end{alineas}

\chapter{Método de pesquisa}\label{cap:metodo_de_pesquisa}

Para atingir-se os objetivos específicos deste projeto serão utilizados os seguintes métodos:
\begin{alineas}
  \item pesquisa bibliográfica em artigos e livros, tendo como base o livro \textit{Compilers: Principles, Techniques, and Tools (2Nd Edition)}~\cite{Aho:2006:CPT:1177220};
  \item análise de tendencias em linguagens de programação existentes, de suas documentações e de suas implementações;
  \item desenvolvimento de uma linguagem de programação;
  \item desenvolvimento de um compilador para a linguagem de programação projetada.
\end{alineas}

\chapter{Planejamento}

\section{Cronograma}\label{cap:cronograma}

O cronograma é descrito pela tabela~\ref{tab:cronograma}.

\begin{table}[h]
  \caption{Planejamento das etapas do trabalho de conclusão de curso}\label{tab:cronograma}
  \resizebox{\textwidth}{!}{
    \begin{tabular}{@{}lcccccccccccccc@{}}
      \toprule
      \multicolumn{1}{c}{Etapas} & \multicolumn{13}{c}{Messes} \\
      \cmidrule(lr){2-15}
      & \multicolumn{2}{c}{2019} & \multicolumn{12}{c}{2020} \\
      \cmidrule(lr){2-3} \cmidrule(lr){4-15}
      & nov & dez & jan & fev & mar & abr & mai & jun & jul & ago & set & out & nov & dez \\
      \midrule
      Entrega da proposta completa
      &04/11&&&&&&&&&&&&&\\
      Estudo da fundamentação teórica
      &\ding{55}&\ding{55}&\ding{55}&\ding{55}&\ding{55}&\ding{55}&&&&&&&&\\
      Desenvolvimento da solução
      &&&\ding{55}&\ding{55}&\ding{55}&\ding{55}&\ding{55}&\ding{55}&\ding{55}&\ding{55}&&&\\
      Desenvolvimento do relatório de Projeto I
      &&&&&&\ding{55}&\ding{55}&\ding{55}&&&&&\\
      Entrega do relatório de Projeto I
      &&&&&&&&\ding{55}&&&&&\\
      Redação do rascunho do TCC
      &&&&&&&&&\ding{55}&\ding{55}&&&&\\
      Desenvolvimento do rascunho do TCC
      &&&&&&&&&&\ding{55}&\ding{55}&\ding{55}&&\\
      Entrega do rascunho do TCC
      &&&&&&&&&&&&\ding{55}&&\\
      Preparação da defesa pública
      &&&&&&&&&&&&\ding{55}&\ding{55}&\\
      Defesa pública
      &&&&&&&&&&&&&\ding{55}&\\
      Ajustes no relatório final do TCC
      &&&&&&&&&&&&&\ding{55}&\ding{55}\\
      \bottomrule
    \end{tabular}
  }
\end{table}

\section{Custos}\label{cap:custos}

Não há custos envolvidos.

\section{Recursos humanos}\label{cap:recursos_humanos}

O projeto sera orientado pelo professor Rafael de Santiago e coorientado pelo aluno de mestrado Evandro Chagas Ribeiro da Costa como indicado pela tabela~\ref{tab:envolvidos}.

\begin{table}[h]
  \caption{Envolvidos no projeto}\label{tab:envolvidos}
    \centering
      \begin{tabular}{@{}ll@{}}
        \toprule
        \multicolumn{1}{c}{Nome} & \multicolumn{1}{c}{Função} \\
        \midrule
        Rafael de Santiago & Responsável \\
        Rafael de Santiago & Orientador \\
        Evandro Chagas Ribeiro da Costa & Coorientador \\
        \bottomrule
      \end{tabular}
\end{table}

\section{Comunicação}\label{cap:comunicacao}

A comunicação sera efetuada através de reuniões intermitentes com o orientador e o coorientador.

\section{Riscos}\label{cap:riscos}

\begin{table}[h]
  \caption{Riscos do projeto}\label{tab:riscos}
    \resizebox{\textwidth}{!}{
      \begin{tabular}{@{}lcccp{4cm}p{4cm}@{}}
        \toprule
        \multicolumn{1}{c}{Risco} & Probabilidade & Impacto & Prioridade & \multicolumn{1}{c}{Estratégia de resposta} & \multicolumn{1}{c}{Ações de prevenção} \\
        \midrule
        Perda de dados & baixa & alto & alta & Recuperação de versões anteriores & Utilizar serviços remotos de versionamento \\
        Ficar doente   & alta  & alto & alta & Fazer tratamento médico necessário & Não abusar da saúde e praticar esportes \\
        Computador estragar & baixa & alto & alta & Comprar um novo computador & Desligar em instabilidades de energia \\
        \bottomrule
      \end{tabular}
    }
\end{table}

\postextual{}

\bibliography{bibliography.bib}

\end{document}
