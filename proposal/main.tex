\documentclass[
  12pt,
  openright,
  twoside,
  a4paper,
  english,
  brazil
]{abntex2}

\usepackage[T1]{fontenc}
\usepackage{indentfirst}
\usepackage{pdfpages}
\usepackage[alf]{abntex2cite}
\usepackage{booktabs}
\usepackage{pifont}

\setlength{\parindent}{1.3cm}

\titulo{Implementação de um front end para IR LLVM}

\autor{Bernardo Ferrari Mendonça}
\local{Brasil}
\data{2019}
\orientador{Rafael de Santiago}
\coorientador{Evandro Chagas Ribeiro da Costa}
\instituicao{
  Universidade Federal de Santa Catarina
  \par
  Departamento de Informática e Estatística
  \par
  Ciência da Computação
}

\tipotrabalho{Trabalho de Conclusão de Curso de Graduação}

\preambulo{Proposta de monografia submetida ao Programa
de Graduação em Ciência da Computação
para a obtenção do Grau de Bacharel.}

\begin{document}
\pretextual{}
\hypersetup{pageanchor=false}
\imprimircapa{}
\hypersetup{pageanchor=true}
\imprimirfolhaderosto{}

\begin{folhadeaprovacao}
  \includepdf{folha-de-aprovacao.pdf}
\end{folhadeaprovacao}

\begin{resumo}
Meu resumo.

\vspace{\onelineskip}
\noindent
\textbf{Palavras-chave}: minhas\@. palavras\@. chave.

\end{resumo}

\begin{KeepFromToc}
  \tableofcontents
\end{KeepFromToc}

\textual{}

\chapter{Introdução}\label{cap:introducao}

Esta vai ser minha introdução\cite{dijkstra1968}!

\chapter{Objetivos}\label{cap:objetivos}

O principal objetivo deste trabalho consiste em projetar, implementar e avaliar uma linguagem de programação similar a linguagem de programação c.
Esta implementação de linguagem de programação deve fazer uso do ecossistema llvm, incluindo suas ferramentas para geração de linguagem de representação intermediaria, para otimização de código e para geração de código objeto executável.

\section{Objetivos específicos}

Os objetivos específicos são:
\begin{alineas}
  \item projetar uma linguagem de programação similar a c;
  \item implementar um analisador léxico, um analisador sintático e uma tradução dirigida a sintaxe;
  \item fazer um estudo sobre a linguagem intermediaria llvm;
  \item implementar uma ferramenta para controlar a otimização da linguagem intermediaria llvm. Além de controlar a tradução da mesma para código objeto;
  \item fazer um estudo sobre o processo de ligação de código;
  \item integrar a ferramenta ao processo de ligação de código do ambiente GNU/Linux;
  \item avaliar as decisões de projeto da linguagem de programação projetada quanto a linguagem c.
\end{alineas}

\chapter{Método de pesquisa}\label{cap:metodo_de_pesquisa}

Para atingir-se os objetivos específicos deste projeto serão utilizados os seguintes métodos:
\begin{alineas}
  \item pesquisa bibliográfica em artigos e livros, tendo como base o livro \textit{Compilers: Principles, Techniques, and Tools (2Nd Edition)}~\cite{Aho:2006:CPT:1177220};
  \item análise de tendencias em linguagens de programação existentes, de suas documentações e de suas implementações;
  \item desenvolvimento de uma linguagem de programação;
  \item desenvolvimento de um compilador para a linguagem de programação projetada.
\end{alineas}

\chapter{Planejamento}
\section{Cronograma}\label{cap:cronograma}

O cronograma é descrito pela tabela~\ref{tab:cronograma}.

\begin{table}[h]
  \caption{Planejamento das etapas do trabalho de conclusão de curso}\label{tab:cronograma}
  \resizebox{\textwidth}{!}{
    \begin{tabular}{@{}lcccccccccccccc@{}}
      \toprule
      \multicolumn{1}{c}{Etapas} & \multicolumn{13}{c}{Messes} \\
      \cmidrule(lr){2-15}
      & \multicolumn{2}{c}{2019} & \multicolumn{12}{c}{2020} \\
      \cmidrule(lr){2-3} \cmidrule(lr){4-15}
      & nov & dez & jan & fev & mar & abr & mai & jun & jul & ago & set & out & nov & dez \\
      \midrule
      Entrega da proposta completa
      &04/11&&&&&&&&&&&&&\\
      Estudo da fundamentação teórica
      &\ding{55}&\ding{55}&\ding{55}&\ding{55}&\ding{55}&\ding{55}&&&&&&&&\\
      Desenvolvimento da solução
      &&&\ding{55}&\ding{55}&\ding{55}&\ding{55}&\ding{55}&\ding{55}&\ding{55}&\ding{55}&&&\\
      Desenvolvimento do relatório de Projeto I
      &&&&&&\ding{55}&\ding{55}&\ding{55}&&&&&\\
      Entrega do relatório de Projeto I
      &&&&&&&&\ding{55}&&&&&\\
      Redação do rascunho do TCC
      &&&&&&&&&\ding{55}&\ding{55}&&&&\\
      Desenvolvimento do rascunho do TCC
      &&&&&&&&&&\ding{55}&\ding{55}&\ding{55}&&\\
      Entrega do rascunho do TCC
      &&&&&&&&&&&&\ding{55}&&\\
      Preparação da defesa pública
      &&&&&&&&&&&&\ding{55}&\ding{55}&\\
      Defesa pública
      &&&&&&&&&&&&&\ding{55}&\\
      Ajustes no relatório final do TCC
      &&&&&&&&&&&&&\ding{55}&\ding{55}\\
      \bottomrule
    \end{tabular}
  }
\end{table}

\section{Custos}\label{cap:custos}

Não há custos envolvidos.

\section{Recursos humanos}\label{cap:recursos_humanos}

O projeto sera orientado pelo professor Rafael de Santiago e coorientado pelo aluno de mestrado Evandro Chagas Ribeiro da Costa como indicado pela tabela~\ref{tab:envolvidos}.

\begin{table}[h]
  \caption{Envolvidos no projeto}\label{tab:envolvidos}
    \centering
      \begin{tabular}{@{}ll@{}}
        \toprule
        \multicolumn{1}{c}{Nome} & \multicolumn{1}{c}{Função} \\
        \midrule
        Rafael de Santiago & Responsável \\
        Rafael de Santiago & Orientador \\
        Evandro Chagas Ribeiro da Costa & Coorientador \\
        \bottomrule
      \end{tabular}
\end{table}

\section{Comunicação}\label{cap:comunicacao}

A comunicação sera efetuada através de reuniões intermitentes com o orientador e o coorientador.

\section{Riscos}\label{cap:riscos}

\begin{table}[h]
  \caption{Riscos do projeto}\label{tab:riscos}
    \resizebox{\textwidth}{!}{
      \begin{tabular}{@{}lcccp{4cm}p{4cm}@{}}
        \toprule
        \multicolumn{1}{c}{Risco} & Probabilidade & Impacto & Prioridade & \multicolumn{1}{c}{Estratégia de resposta} & \multicolumn{1}{c}{Ações de prevenção} \\
        \midrule
        Perda de dados & baixa & alto & alta & Recuperação de versões anteriores & Utilizar serviços remotos de versionamento \\
        Ficar doente   & alta  & alto & alta & Fazer tratamento médico necessário & Não abusar da saúde e praticar esportes \\
        Computador estragar & baixa & alto & alta & Comprar um novo computador & Desligar em instabilidades de energia \\
        \bottomrule
      \end{tabular}
    }
\end{table}

\postextual{}

\bibliography{bibliography.bib}

\end{document}
