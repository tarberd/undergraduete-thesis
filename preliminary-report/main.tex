%dica: use a opção oneside se houver um limite (e.g., 20) de páginas
\documentclass[embeddedlogo]{ufsc-thesis-rn46-2019}

\usepackage[T1]{fontenc} % fontes
\usepackage[utf8]{inputenc} % UTF-8
\usepackage{lipsum} % Gerador de texto
\usepackage{pdfpages} % Inclui PDF externo (ficha catalográfica)

\graphicspath{{./dir/}}

%%%%%%%%%%%%%%%%%%%%%%%%%%%%%%%%%%%%%%%%%%%%%%%%%%%%%%%%%%%%%%%%%%%%
%%% Configurações da classe (dados do trabalho)                  %%%
%%%%%%%%%%%%%%%%%%%%%%%%%%%%%%%%%%%%%%%%%%%%%%%%%%%%%%%%%%%%%%%%%%%%

% Preâmbulo
\titulo{Template \LaTeX{} seguindo a RN 46/2019/CPG da UFSC}
\autor{Omar Ravenhurst}
% Importante! Para documentos em inglês, não use today, digite a data em
% pt_BR, como deve aparecer na folha de certificação.
\data{1 de Agosto de 2019}
\instituicao{Universidade Federal de Santa Catarina}
\centro{Centro Tecnológico}
\programa{Programa de Pós-Graduação em Ciência da Computação}
\tese % ou \dissertacao
\local{Florianópolis} % Apenas cidade! Sem estado
% template da BU usa doutor/mestre em minúsculo, Bacharel/Licenciado em Title case.
\titulode{doutor em Ciência da Computação}

%%% Atenção! No caso de TCC, além de usar \tcc, outros comandos devem ser fornecidos:
%%%
% \tcc
% \departamento{Departamento de Informática e Estatística}
% \curso{Ciência da Computação}
% \titulode{Bacharel em Ciência da Computação}
% %% Para TCCs, orientadores e coorientadores podem ser externos, logo a
% %% BU exige que sua afiliação seja explicitada. Por padrão, assume-se
% %% UFSC. Você pode alterar a afiliação com os comandos abaixo:
% \afiliacaoorientador{Universidade Federal de Santa Catarina}
% \afiliacaocoorientador{Universidade Federal da Terra de Ninguém}

% Orientador, coorientador, membros da banca e coordenador
% As regras da BU agora exigem que Dr. apareça **depois** do nome
% Dica: para gerar Profᵃ. use Prof\textsuperscript{a}.
% Dica 2: para feminino use \orientadora e \coorientadora
\orientador{Prof. Ben Trovato, Dr.}
\coorientador{Prof. Lars Thørväld, Dr.}
\membrobanca{Prof. Valerie Béranger, Dr.}{Universidade Federal de Santa Catarina}
\membrobanca{Prof. Mordecai Malignatus, Dr.}{Universidade Federal de Santa Catarina}
\membrobanca{Prof. Huifen Chan, Dr.}{Universidade Federal de Santa Catarina}
% Dica: se feminino, \coordenadora
\coordenador{Prof. Charles Palmer, Dr}

\begin{document}

%%%%%%%%%%%%%%%%%%%%%%%%%%%%%%%%%%%%%%%%%%%%%%%%%%%%%%%%%%%%%%%%%%%%
%%% Principais elementos pré-textuais                            %%%
%%%%%%%%%%%%%%%%%%%%%%%%%%%%%%%%%%%%%%%%%%%%%%%%%%%%%%%%%%%%%%%%%%%%

% Inicia parte pré-textual do documento capa, folha de rosto, folha de
% aprovação, aprovação, resumo, lista de tabelas, lista de figuras, etc.
\pretextual%
\imprimircapa%
\imprimirfolhaderosto*
\protect\incluirfichacatalografica{ficha.pdf}
\imprimirfolhadecertificacao

\begin{dedicatoria}
  Este trabalho é dedicado à wikipedia e ao stackoverflow.
\end{dedicatoria}

\begin{agradecimentos}
  O presente trabalho foi realizado com apoio da Coordenação de Aperfeiçoamento de Pessoal de Nível Superior -- Brasil (CAPES) -- Código de Financiamento 001.
\end{agradecimentos}

\begin{epigrafe}
  For a number of years I have been familiar with the observation that the quality of programmers is a decreasing function of the density of go to statements in the programs they produce \\
  \cite{dijkstra1968}
\end{epigrafe}


\begin{resumo}[Resumo]
  Aqui deve ser inserido um resumo de 150 a 500 palavras (limitação de tamanho dada pela BU). A linguagem deve ser português e a hifenização já foi alterada. O resumo em português deve preceder o resumo em inglês, mesmo que o trabalho seja escrito em inglês. A BU também diz que deve ser usada a voz ativa e o discurso deve ser na 3ª pessoa. A estrutura do resumo pode ser similar a estrutura usada em artigos: Contexto -- Problema -- Estado da arte -- Solução proposta  -- Resultados.

  % Atenção! a BU exige separação através de ponto (.). Ela recomanda de 3 a 5 keywords
  \vspace{\baselineskip}
  \textbf{Palavras-chave:} Palavra-chave. Ponto como separador. Bla.
\end{resumo}


\begin{resumo}[Resumo Estendido]
  \section*{Introdução}
  A hifenização é alterada para \texttt{brazil}, mesmo para documentos em inglês. Descrever brevemente esses itens exigidos pela BU. Como a RN 95/CUn/2017 é mais recente e impõe outras regras a revelia de regimentos e regulamentos, é mais sábio obedecê-la. Lembre que esse resumo estendido deve term entre 2 e 5 páginas.

  \lipsum[1]
  \section*{Objetivos}
  \lipsum[2]
  \section*{Metodologia}
  \lipsum[3]
  \section*{Resultados e Discussão}
  \lipsum[4]
  \section*{Considerações Finais}
  \lipsum[5]

  \vspace{\baselineskip}  % Atenção! manter igual ao resumo
  \textbf{Palavras-chave:} Palavra-chave. Outra Palavra-chave composta. Bla.
\end{resumo}


\begin{abstract}
  Enlish version of the plain ``resumo'' above. Done with environment
  \texttt{abstract}. Hyphenization is automatically changed to english.

  \vspace{\baselineskip}
  \textbf{Keywords:} Keyword. Another Compound Keyword. Bla.
\end{abstract}

\listoffigures*

% Lista para ambiente algorithm
\listofalgorithms*

\begin{listadesimbolos}
  $\gets$   & Atribuição \\
  $\exists$   & Quantificação existencial \\
  $\rightarrow$   & Implicação \\
  $\wedge$   & E lógico \\
  $\vee$   & Ou lógico \\
  $\neg$   & Negação lógica \\
  $\mapsto$   & Mapeia para \\
  $\sqsubseteq$   & Subclasse (em ontologias) \\
  $\subseteq$   & Subconjunto: $\forall x\;.\; x \in A \rightarrow x \in B$ \\
  $\langle\ldots\rangle$ & Tupla \\
  $\forall$   & Quantificação universal \\
  mmmmm & Nenhum sentido, apenas estou aqui para demonstrar a largura máxima dessas colunas. Ao abrir o ambiente \texttt{listadesimbolos}, pode-se fornecer um argumento opcional indicando a largura da coluna da esquerda (o default é de 5em): \texttt{\textbackslash{}begin\{listadesimbolos\}[2cm] .... \textbackslash{}end\{listadesimbolos\}} \\
\end{listadesimbolos}

\tableofcontents*%

%%%%%%%%%%%%%%%%%%%%%%%%%%%%%%%%%%%%%%%%%%%%%%%%%%%%%%%%%%%%%%%%%%%%
%%% Corpo do texto                                               %%%
%%%%%%%%%%%%%%%%%%%%%%%%%%%%%%%%%%%%%%%%%%%%%%%%%%%%%%%%%%%%%%%%%%%%
\textual%

\chapter{Introdução}
\label{ch:intro}

Bem-vindo ao template no overleaf classe \texttt{ufsc-thesis-rn46-2019}. Essa
classe é um conjunto de customizações aplicadas à classe
\href{https://ctan.org/pkg/abntex2}{\abnTeX} e ao pacote \texttt{abntex2cite}.
O objetivo da classe \texttt{ufsc-thesis-rn46-2019} é simplório: adequar o
\abnTeX{} às \href{http://portal.bu.ufsc.br/normalizacao/}{normas emitidas pela
Biblioteca Universitária da UFSC} em sequência à
\href{https://repositorio.ufsc.br/handle/123456789/197121}{Resolução Normativa
nº 46/2019/CPG}.


\section{Autores, suporte e atualizações}

Essa classe foi escrita inicialmente por dois alunos do
\href{http://ppgcc.posgrad.ufsc.br/}{PPGCC da UFSC}:
\href{mailto:alexishuf@gmail.com}{Alexis Huf} e
\href{mailto:gustavo.zambonin@posgrad.ufsc.br}{Gustavo Zambonin}.
Há o risco de esse arquivo não ser atualizado a cada \textit{pull request},
então confira a lista de mártires no GitHub. Essa classe é mantida no
repositório
\href{https://github.com/alexishuf/ufsc-thesis-rn46-2019/}{alexishuf/ufsc-thesis-rn46-2019}.
Atualizações podem ser encontradas nesse repositório. \textit{Issues} e PRs são
bem vindos.

\section{Exemplos de formatação}
\label{sec:ex}

Essa frase é verdadeira pois tem um \texttt{cite} no final \cite{turing1937}. Essa
é mais verdadeira ainda pois tem um  \texttt{cite} duplo no
final \cite{turing1937,dijkstra1968}. Já esta frase inofensiva usa
 \texttt{citeonline} para citar \citeonline{dijkstra1968}
nominalmente. O trabalho de \citeonline{diffie1976} foi altamente influente
\cite{diffie1976}. Essa outra frase cita o trabalho que \citeonline{Saleem2018}
escreveu com outros 4 autores. Para algo completamente novo, veja um footnote
com url\footnote{\url{http://example.org/}}

Mais algumas citações de tipos específicos de documentos:
\begin{itemize}
\item @inproceedings: \citeonline{Ullman1989magic}, jabuti
  \cite{Ullman1989magic};
\item @article: \citeonline{Distefano2019}, framboesa \cite{Distefano2019};
\item @book: \citeonline{Abiteboul1995}, goiaba \cite{Abiteboul1995};
\item @incollection: \citeonline{Forgy1989}, melancia \cite{Forgy1989};
\item @techreport: \citeonline{rdf11}, figo \cite{rdf11}.
\end{itemize}

A lista abaixo mostra o efeito de  \texttt{autoref} com capítulos e (sub)seções.

\begin{itemize}
\item Há coisas no \autoref{ch:intro};
\item Há coisas na \autoref{sec:stuff};
\item Há coisas na \autoref{sec:other};
\item Há coisas na \autoref{sec:yet-another} (\abnTeX{} come um ``sub'' intencionalmente).
\end{itemize}

Citações são feitas com o ambiente \texttt{citacao}. A BU faz
as mesmas exigências que já são o \textit{default} na classe
\abnTeX\footnote{O alinhamento e o filete de notas de rodapé também
não necessitou de modificações, além do tamaho da fonte. Essa frase
não serve a nenhum propósito além de causar uma quebra de linha para
que o alinhamento seja avaliado.}.

\begin{citacao}
  A elaboração do trabalho de conclusão de curso em nível de mestrado
  e de doutorado na UFSC deverá atender aos critérios e procedimentos
  estabelecidos nesta resolução normativa e em diretrizes
  estabelecidas pela Pró-Reitoria de Pós-Graduação e pelos Programas
  de Pós-Graduação.
\end{citacao}

Atenção! O template da BU deixa figuras e tabelas alinhadas à esquerda. No
entanto, o tutorial de Word disponibilizado pela BU diz que legendas e
\emph{captions} devem respeitar o ``alinhamento da ilustração'' (e apresenta
uma ilustração alinhada à esquerda). O tutorial explicando a ABNT mostra uma
figura centralizada com legendas alinhadas a esquerda e com recuo até o começo
da figura. O autor do \texttt{.cls} se exime de qualquer culpa.
Veja na \autoref{fig:logo} o efeito de se usar \texttt{centering}.

\begin{figure}[t]
  \centering
  \caption{Logotipo da Universidade Federal de Santa Catarina.}
  \label{fig:logo}

  \includegraphics[width=.2\linewidth]{\jobname-logo.pdf}
  \fonte{O autor.}
\end{figure}

Algoritmos podem ser incluidos no ambiente \texttt{algorithm}. Atenção! Esse
ambiente é apenas uma classe de \emph{float}. Logo esse ambiente apenas oferece
a funcionalidade de colocar legendas e numerar algoritmos que serão exibidos em
uma lista dedicada com \texttt{\textbackslash{}listofalgorithms*}. Para
efetivamente escrever os lagoritmos considere pacotes como
\texttt{algorithmicx}, \texttt{algorithm2e} ou \texttt{minted}.

\begin{algorithm}
  \centering
  \caption{Exemplo sem sentido algum.}

  \fbox{
    \parbox{.6\textwidth}{
      \begin{itemize}
      \item Não use itemize como algoritmo!
      \item Escolha um pacote como \texttt{algorithmicx}, \texttt{algorithm2e} ou \texttt{minted}.
      \end{itemize}
    }
  }

  \fonte{O autor.}
\end{algorithm}

\subsection{Coisas}
\label{sec:stuff}
Imagine alguma afirmação de alto valor científico aqui.

\subsubsection{Outras coisas mais}
\label{sec:other}
Estudos demonstram que essa afirmação é falsa.

\subsubsubsection{Ainda outras coisas mais}
\label{sec:yet-another}
Fazer a grama verde, como? Novamente o jogo foi perdido. Opcionalmente, tudo
pode ser opcional. Recursos foram gastos com isso. Descubra a verdade nas
capitalizadas.
% Fiquei 15 minutos mais próximo da morte ao escrever isso. Você pode chegar
% ainda mais perto se tentar entender.

%%%%%%%%%%%%%%%%%%%%%%%%%%%%%%%%%%%%%%%%%%%%%%%%%%%%%%%%%%%%%%%%%%%%
%%% Elementos pós-textuais                                       %%%
%%%%%%%%%%%%%%%%%%%%%%%%%%%%%%%%%%%%%%%%%%%%%%%%%%%%%%%%%%%%%%%%%%%%

\postextual
\bibliography{example}

\apendices

\chapter{Exemplo de Apêndice}~\label{ch:apendice}

% Uso de \cite em apêndice/anexo como se fossem antes do \bibliography{}.  NBR
% 14724 e NBR 6023, assim como os documentos da BU não especificam nada sobre
% citações dentro de apêndices/anexos. No entanto, em email trocado com a BU, a
% orientação foi de usar \cite{} normalmente e deixar que as referências sejam
% listadas na única bibliografia do documento, mesmo que esta esteja antes dos
% apêndices. A argumentação é que apêndices e anexos são numerados e fazem parte
% do documento, logo suas referências devem ser listadas como referências do
% documento. Além disso as normas não prevem segmentar as referências por
% capítulos.
~\cite{turing1937} \lipsum[1]


\end{document}

